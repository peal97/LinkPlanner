% !TEX root = ../../main_netxpto.tex
\clearpage

\section{IIR Filter}

\begin{refsection}

\begin{tcolorbox}	
	\begin{tabular}{p{2.75cm} p{0.2cm} p{10.5cm}} 	
		\textbf{Header File}   &:& iir\_filter\_*.h \\
		\textbf{Source File}   &:& iir\_filter\_*.cpp \\
		\textbf{Version}	   &:& 20180718 (Andoni Santos)
	\end{tabular}
\end{tcolorbox}

\subsection*{Input Parameters}

\begin{table}[H]
	\centering
	\begin{tabular}{|l|l|l|}
		\hline
		\textbf{Name}  		 & \textbf{Type}  & \textbf{Default Value}    	\\\hline
%		Confidence     		 & double         & 0.95              	\\\hline
%		LowestMinorant & double         & $1\times10^{-10}$ \\ \hline
	\end{tabular}
\end{table}


\subsection*{Methods}
IIR\_Filter() {}
\bigbreak
IIR\_Filter(vector$<$Signal *$>$ \&InputSig, vector$<$Signal *$>$ \&OutputSig){}
\bigbreak
void initialize(void)
\bigbreak
bool runBlock(void)
\bigbreak
void setBCoeff(vector<double> newBCoeff)
\bigbreak
void setACoeff(vector<double> newACoeff)
\bigbreak
int getFilterOrder(void)

\subsection*{Input Signals}

\textbf{Number}: 1 or 2\\
\textbf{Type}: OpticalSignal or TimeContinuousAmplitudeContinuousReal

\subsection*{Output Signals}

\textbf{Number}: 1 or 2\\
\textbf{Type}: OpticalSignal or TimeContinuousAmplitudeContinuousReal

\subsection*{Functional Description}
This method implements Infinite Impulse Response Filters. Currently it does so by Canonic Realization~\cite{jeruchim06}.

\subsection*{Theoretical Description}

\subsection*{Known Issues}

%%%%%%%%%%%%%%%%%%%%%%%%%%%%%%%%%%%%%%%%%%%%%%%%%%%%%%%%%%%%%%%%%%%%%%%%%%%%%%%%%%%%%%%%%%%%%%%%%%%%%%%%%%%%
% References
%%%%%%%%%%%%%%%%%%%%%%%%%%%%%%%%%%%%%%%%%%%%%%%%%%%%%%%%%%%%%%%%%%%%%%%%%%%%%%%%%%%%%%%%%%%%%%%%%%%%%%%%%%%%


% bibliographic references for the section ----------------------------
\clearpage
\printbibliography[heading=subbibliography]
\end{refsection}
\addcontentsline{toc}{subsection}{Bibliography}
\cleardoublepage
% ---------------------------------------------------------------------
