% !TEX root = ../../main_netxpto.tex
\clearpage

\section{SNR Estimator}

\begin{refsection}

\begin{tcolorbox}	
	\begin{tabular}{p{2.75cm} p{0.2cm} p{10.5cm}} 	
		\textbf{Header File}   &:& snr\_estimator\_*.h \\
		\textbf{Source File}   &:& snr\_estimator\_*.cpp \\
		\textbf{Version}	   &:& 20180227 (Andoni Santos)
	\end{tabular}
\end{tcolorbox}

\subsection*{Input Parameters}

\begin{table}[H]
	\centering
	\begin{tabular}{|l|l|l|}
		\hline
		\textbf{Name}  		 & \textbf{Type}  & \textbf{Value}    	\\\hline
		Confidence     		 & double         & 0.95              	\\\hline
		measuredIntervalSize & int 			  & 1024				\\\hline
		windowType			 & WindowType     & Hamming			  	\\\hline
		segmentSize			 & int			  & 512					\\\hline
		overlapCount  		 & int			  & 256					\\\hline
%		LowestMinorant & double         & $1\times10^{-10}$ \\ \hline
	\end{tabular}
\end{table}


\subsection*{Methods}

\subsection*{Input Signals}

\textbf{Number}: 1 or 2\\
\textbf{Type}: OpticalSignal or TimeContinuousAmplitudeContinuousReal

\subsection*{Output Signals}

\textbf{Number}: 1 or 2\\
\textbf{Type}: OpticalSignal or TimeContinuousAmplitudeContinuousReal

\subsection*{Functional Description}
This  accepts one OpticalSignal or TimeContinuousAmplitudeContinousReal signal
(or two, an In-phase signal and a quadrature signal), estimates the
signal-to-noise ratio for a given time interval and outputs an exact copy of the
input signal. The estimated SNR value is printed to \textit{cout} after each
calculation.  The block also writes a \textit{.txt} file reporting the estimated
signal-to-noise ratio, a count of the number of measurements and the
corresponding bounds for a given confidence level, based on measurements made on
consecutive signal segments.


\subsection*{Theoretical Description}\label{snrcalc}
This block can use on of several methods to estimate the signal's SNR. Any of
them can be usefull for particular use cases.

Currently there are three methods implemented:

\begin{itemize}
	\item \textbf{powerSpectrum} - Spectral
		analysis~\cite{harris12,xiao10,kashefi12};
	\item \textbf{m2m4} - Moments method~\cite{matzner93};
	\item \textbf{ren} - Modified moments method~\cite{ren05};
\end{itemize}

\subsubsection*{Spectral analysis}
Several methods exist for SNR estimation, most of them requiring prior
knowledge of the sample time or of the signal's conditions. Spectrum-based SNR
estimators have been shown to provide accurate results even these informations
are not available, at the cost of being less efficient ~\cite{harris12,xiao10,kashefi12}. 


The power spectrum is estimated through Welch's periodogram over a predefined
interval size~\cite{john2007digital}. This averages the values in each bin,
providing a smoother estimate of the power spectrum.
Using the power spectrum obtained from this sequence, the frequency interval
containing the signal is estimated from the sampling rate, symbol rate and
modulation type. The power contained in this interval will include both the signal
and the white noise in those frequencies.

The power of a signal can be calculated from its power spectrum~\cite{john2007digital}:

\begin{equation}
P = \frac{1}{N} \sum_{n=0}^{N-1} {|x(n)|}^2 = \frac{1}{N^2} \sum_{k=0}^{N-1} {|X(k)|}^2
\end{equation}


By default, the noise is assumed to be AWGN with a given constant spectral
density. Therefore, by estimating the noise spectral density, an estimation can
be made of the noise's full power. The noise power spectral density is therefore
estimated from the the spectrum in the area without signal. This is the used to
calculate the total noise power. The signal power is obtained by summing the FFT
bins within the signal's frequency interval and subtracting the corresponding superimposed noise, according to what was previously estimated. The power SNR is the ratio between these two values.

\begin{equation}
	\text{SNR} = \frac{P_{s}}{P_n} = \frac{P_{s+n} - P_n}{P_n} = \frac{P_{s+n - n_0} R_{sym}}{n_0 R_{sam}}
\end{equation}

This value is saved and the process is repeated for every sequence of samples.
The confidence interval is calculated based on the all the obtained SNR
values~\cite{tranter2004principles}. The SNR value and confidence interval are
then saved to a text file.

\subsubsection*{Moments method}\label{sec:snrEstm2m4}
If the sampling time is known, or it the signal has already been sampled, it is
possible to resort to higher order statistics to get an SNR estimation. One
popular method, due to its simplicity and efficiency, uses the second and fourth
order moments of the signal wo estimate the SNR~\cite{matzner93}.

Let the second and fourth order moments be $M_2$ and $M_4$. In addition, let
$S_k$ be the sampled signal, with $S_{I_k}$ and $S_{Q_k}$ as its in-phase and
quadrature components.

\begin{equation}
	M_2 = E\{|S_k|^2\} = E\{S_{i_k}^2 + S_{Q_k}^2\} = P_S + P_N
\end{equation}

\begin{equation}
	M_4 = E\{|S_k|^4\} = E\{(S_{i_k}^2 + S_{Q_k}^2)^2\} = k_e P_S^2 + 4P_S P_N + k_n P_N^2
\end{equation}

Here, $k_e$ and $k_n$ are constants depending on the properties of the
probability density function of the signal and noise processes. In practice,
they are known from the modulation scheme used and the noise characteristics.
For signals with constant energy, for instance, $k_e = 1$, and for two
dimensional gaussian noise $k_n = 2$.
It follows that the signal and noise powers can be defined as follows: 

\begin{equation}
	P_S = \sqrt{2 M_2^2 - M_ 4}
\end{equation}

\begin{equation}
	P_N = M_2 - \sqrt{2 M_2^2 - M_ 4}
\end{equation}

and therefore

\begin{equation}
	\text{SNR} = \frac{\sqrt{2EM_2^2 - M_4}}{M_2 - \sqrt{2M_2^2 - M_4}}
\end{equation}

\subsubsection*{Modified Moments method}

\indent This method works like the previous one, but tries to decrease bias and mean
square error by using a different fourth moment of the received
signal~\cite{ren05}.

\begin{equation}
	M^\prime_{4} = E\{(S^2_{I_k} - S^2_{Q_k})^2\} = 2 P_S P_N + P_N^2
\end{equation}

Thus, $P_S$ ,$P_N$ and the SNR become:

\begin{equation}
	P^\prime_S = \sqrt{M^2_2 - M^\prime_4}
\end{equation}

\begin{equation}
	P^\prime_N = M_2 \sqrt{M^2_2 - M^\prime_4}
\end{equation}

\begin{equation}
	\text{SNR} = \frac{\sqrt{M_2^2 - M^\prime_4}}{M_2 - \sqrt{M_2^2 - M^\prime_4}}
\end{equation}

%This block output signal is exactly equal to the input signal, to estimate SNR at any point in a given simulated system without interfering with it.

%The $\widehat{\text{BER}}$ is obtained by counting both the total number received bits, $N_T$, and the number of coincidences, $K$, and calculating their relative ratio:
%\begin{equation}
%\widehat{\text{BER}}=1-\frac{K}{N_T}.
%\end{equation}

%The upper and lower bounds, $\text{SNR}_\text{UB}$ and $\text{SNR}_\text{LB}$ respectively, are calculated using the Clopper-Pearson confidence interval.
%, which returns the following simplified expression for $N_T>40$~\cite{almeida2016continuous}:
%\begin{align}
%	\text{BER}_\text{UB}&=\widehat{\text{BER}}+\frac{1}{\sqrt{N_T}}z_{\alpha/2}\sqrt{\widehat{\text{BER}}(1-\widehat{\text{BER}})}+\frac{1}{3N_T}\left[2\left(\frac{1}{2}-\widehat{\text{BER}}\right)z_{\alpha/2}^2+(2-\widehat{\text{BER}})\right]\\
%	\text{BER}_\text{LB}&=\widehat{\text{BER}}-\frac{1}{\sqrt{N_T}}z_{\alpha/2}\sqrt{\widehat{\text{BER}}(1-\widehat{\text{BER}})}+\frac{1}{3N_T}\left[2\left(\frac{1}{2}-\widehat{\text{BER}}\right)z_{\alpha/2}^2-(1+\widehat{\text{BER}})\right],
%\end{align}
%where $z_{\alpha/2}$ is the $100\left(1-\frac{\alpha}{2}\right)$th percentile of a standard normal distribution.

\subsection*{Known Issues}\label{snrestissues}

This block currently only works with either one or two
\textit{TimeContinuousAmplitudeContinuousReal} signals, depending on the chosen
method. It has also only been tested with the in-phase and quadrature components of an MQAM signal with M=4.
If the noise's power spectral density is high enough for the signal not to be
detected, the block will fail (in the mentioned case, typically
values of SNR < -10dB can be unreliable).



%%%%%%%%%%%%%%%%%%%%%%%%%%%%%%%%%%%%%%%%%%%%%%%%%%%%%%%%%%%%%%%%%%%%%%%%%%%%%%%%%%%%%%%%%%%%%%%%%%%%%%%%%%%%
% References
%%%%%%%%%%%%%%%%%%%%%%%%%%%%%%%%%%%%%%%%%%%%%%%%%%%%%%%%%%%%%%%%%%%%%%%%%%%%%%%%%%%%%%%%%%%%%%%%%%%%%%%%%%%%


% bibliographic references for the section ----------------------------
\clearpage
\printbibliography[heading=subbibliography]
\end{refsection}
\addcontentsline{toc}{subsection}{Bibliography}
\cleardoublepage
% ---------------------------------------------------------------------
